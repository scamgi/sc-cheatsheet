%%%%%%%%%%%%%%%%%%%%%%%%%%%%%%%%%%%%%%%%%%%%%%%%%%%%%%%%
% preamble.tex - Pacchetti e impostazioni
%%%%%%%%%%%%%%%%%%%%%%%%%%%%%%%%%%%%%%%%%%%%%%%%%%%%%%%%

% --- Lingua e codifica ---
\usepackage[utf8]{inputenc}
\usepackage[T1]{fontenc}
\usepackage[italian]{babel}

% --- Matematica ---
\usepackage{amsmath}
\usepackage{amssymb}
\usepackage{amsfonts}
\usepackage{esvect} % Per vettori più carini \vv{}

% --- Grafica e Colori ---
\usepackage{graphicx}
\usepackage{xcolor}
\usepackage{pgfplots} % Per disegnare i grafici

% --- Layout della pagina ---
\usepackage[a4paper, top=2cm, bottom=2cm, left=1.5cm, right=1.5cm]{geometry} % Margini
\usepackage{fancyhdr} % Per header e footer personalizzati
\usepackage{paracol} % Per layout a colonne
\usepackage{multicol} % Per avere più colonne

% --- Tabelle e liste ---
\usepackage{booktabs} % Per tabelle più professionali
\usepackage{enumitem} % Per personalizzare le liste


% --- Impostazioni di Header e Footer ---
\pagestyle{fancy}
\fancyhf{} % Pulisce tutti i campi di header e footer
\renewcommand{\headrulewidth}{0pt} % Rimuove la linea dell'header
\cfoot{\thepage} % Numero di pagina centrato nel footer

% --- Comandi personalizzati ---
\newcommand{\rect}{\text{rect}}
\newcommand{\tri}{\text{tri}}
\newcommand{\sinc}{\text{sinc}}
\newcommand{\FT}{\mathcal{F}} % Trasformata di Fourier
\newcommand{\IFT}{\mathcal{F}^{-1}} % Anti-trasformata
\newcommand{\E}{\mathbb{E}} % Operatore di Valore Atteso
\newcommand{\var}{\text{Var}} % Varianza
\newcommand{\cov}{\text{Cov}} % Covarianza

% --- Hyperlinks and PDF metadata ---
\usepackage{hyperref}
\hypersetup{
  colorlinks=true,      % Colora i link invece di usare box
  linkcolor=black,       % Colore per i link interni (es. indice)
  urlcolor=magenta,     % Colore per gli URL esterni
  citecolor=green,      % Colore per le citazioni
  pdftitle={Cheatsheet di Segnali per le Comunicazioni}, % Titolo del PDF
  pdfauthor={Giacomo Scampini}, % Autore del PDF
  pdfsubject={Formulario per l'esame di Segnali per le Comunicazioni},
  pdfkeywords={Segnali, Comunicazioni, Politecnico di Milano, Formulario}
}

% --- CONFIGURAZIONE PGFPLOTS ---
\pgfplotsset{compat=1.18} % Imposta la versione di pgfplots per la compatibilità
\usepgfplotslibrary{fillbetween}

% --- Teoremi e definizioni (opzionale, per un look più pulito) ---
\usepackage{amsthm}
\newtheoremstyle{mystyle}% name
  {}% space before
  {}% space after
  {\itshape}% body font
  {}% indent
  {\bfseries}% head font
  {.}% punctuation
  {.5em}% space after head
  {}% head spec
\theoremstyle{mystyle}
\newtheorem{theorem}{Teorema}
\newtheorem{definition}{Definizione}