%%%%%%%%%%%%%%%%%%%%%%%%%%%%%%%%%%%%%%%%%%%%%%%%%%%%%%%%
% chapters/ese1.tex
%%%%%%%%%%%%%%%%%%%%%%%%%%%%%%%%%%%%%%%%%%%%%%%%%%%%%%%%
\chapter{Esercizio 1: Analisi di Segnali}
\addcontentsline{toc}{chapter}{Esercizio 1: Analisi di Segnali}

\section{Strategia Generale}
L'esercizio 1 richiede tipicamente di calcolare la trasformata di Fourier $X(f)$ di un segnale $x(t)$, tracciarne i grafici di modulo e fase e, a volte, calcolare un'uscita $y(t)$ data da una convoluzione o un prodotto.

\subsection{A) Calcolo della Trasformata di Fourier $X(f)$}
\begin{enumerate}
    \item \textbf{Scomporre il segnale:} Identificare le funzioni elementari ($\rect, \tri, \cos, \dots$) e le operazioni ($\cdot, *, \text{traslazione}, \dots$) che compongono $x(t)$.
    \item \textbf{Applicare le proprietà:} Usare le proprietà della FT per trasformare le operazioni nel dominio del tempo in operazioni nel dominio della frequenza.
    \begin{itemize}
        \item Se $x(t) = v(t-t_0)$, prima calcola $V(f)$ e poi applica la proprietà di traslazione: $X(f) = V(f)e^{-j2\pi ft_0}$.
        \item Se $x(t) = v(t)\cos(2\pi f_0 t)$, prima calcola $V(f)$ e poi applica la modulazione: $X(f) = \frac{1}{2}[V(f-f_0) + V(f+f_0)]$.
        \item Se $x(t) = a(t) * b(t)$, allora $X(f) = A(f)B(f)$.
        \item Se $x(t) = a(t) \cdot b(t)$, allora $X(f) = A(f) * B(f)$.
    \end{itemize}
    \item \textbf{Esempio comune:} La funzione $\tri\left(\frac{f}{W}\right)$ è la convoluzione di due rettangoli: $\frac{1}{W} \rect\left(\frac{f}{W}\right) * \frac{1}{W} \rect\left(\frac{f}{W}\right)$. Questo deriva dal fatto che nel tempo $\sinc^2(Wt)$ è il prodotto di due $\sinc(Wt)$.
\end{enumerate}

\subsection{B) Grafici di Modulo e Fase}
Dato $X(f) = A(f) e^{j\phi(f)}$, dove $A(f)$ è reale e positivo:
\begin{itemize}
    \item \textbf{Modulo:} $|X(f)| = A(f)$
    \item \textbf{Fase:} $\angle X(f) = \phi(f)$
\end{itemize}
\textbf{Attenzione:} Un fattore di traslazione temporale $e^{-j2\pi f t_0}$ introduce una fase lineare.
\begin{itemize}
    \item Modulo: $|X(f)e^{-j2\pi f t_0}| = |X(f)|$
    \item Fase: $\angle (X(f)e^{-j2\pi f t_0}) = \angle X(f) - 2\pi f t_0$
\end{itemize}
Se $X(f)$ è reale ma non sempre positivo, la fase sarà $0$ dove $X(f) > 0$ e $\pi$ (o $-\pi$) dove $X(f) < 0$.

\subsection{C) Calcolo dell'uscita $y(t)$}
Se $y(t) = x(t) * h(t)$, il modo più semplice è quasi sempre passare in frequenza:
\begin{enumerate}
    \item Calcola $X(f)$ e $H(f)$.
    \item Calcola $Y(f) = X(f)H(f)$.
    \item Antitrasforma $Y(f)$ per trovare $y(t)$.
\end{enumerate}
Se l'ingresso $x(t)$ è un coseno, $x(t) = \cos(2\pi f_0 t)$, l'uscita sarà:
$$ y(t) = |H(f_0)| \cos(2\pi f_0 t + \angle H(f_0)) $$