%%%%%%%%%%%%%%%%%%%%%%%%%%%%%%%%%%%%%%%%%%%%%%%%%%%%%%%%
% chapters/formule_generali.tex
%%%%%%%%%%%%%%%%%%%%%%%%%%%%%%%%%%%%%%%%%%%%%%%%%%%%%%%%
\chapter*{Formule Generali}
\addcontentsline{toc}{chapter}{Formule Generali}

\section{Formule Trigonometriche}
Una raccolta di identità trigonometriche fondamentali per la manipolazione di segnali sinusoidali.

\subsection{Relazioni di Eulero}
Le formule di Eulero sono essenziali per passare dalla notazione trigonometrica a quella esponenziale complessa, semplificando i calcoli di modulazione e le trasformate di Fourier.
\begin{align*}
    \cos(\theta) &= \frac{e^{j\theta} + e^{-j\theta}}{2} \\
    \sin(\theta) &= \frac{e^{j\theta} - e^{-j\theta}}{2j}
\end{align*}

\subsection{Formule di Addizione e Sottrazione}
Queste formule sono utili per scomporre o combinare fasi e frequenze.
\begin{align*}
    \cos(\alpha \pm \beta) &= \cos(\alpha)\cos(\beta) \mp \sin(\alpha)\sin(\beta) \\
    \sin(\alpha \pm \beta) &= \sin(\alpha)\cos(\beta) \pm \cos(\alpha)\sin(\beta)
\end{align*}

\subsection{Formule di Prodotto-Somma (Formule di Werner)}
Trasformano il prodotto di sinusoidi (che nel dominio della frequenza è una convoluzione) in una somma di sinusoidi (che è una somma di impulsi).
\begin{align*}
    \cos(\alpha)\cos(\beta) &= \frac{1}{2}[\cos(\alpha - \beta) + \cos(\alpha + \beta)] \\
    \sin(\alpha)\sin(\beta) &= \frac{1}{2}[\cos(\alpha - \beta) - \cos(\alpha + \beta)] \\
    \sin(\alpha)\cos(\beta) &= \frac{1}{2}[\sin(\alpha + \beta) + \sin(\alpha - \beta)]
\end{align*}

\subsection{Formule di Potenza (Angoli Multipli)}
Utili per linearizzare espressioni con potenze di seni e coseni, che spesso emergono da operazioni non lineari.
\begin{align*}
    \cos^2(\theta) &= \frac{1 + \cos(2\theta)}{2} \\
    \sin^2(\theta) &= \frac{1 - \cos(2\theta)}{2}
\end{align*}

\subsection{Proprietà di Parità e Periodicità}
\begin{itemize}
    \item \textbf{Coseno (pari):} $\cos(-\theta) = \cos(\theta)$
    \item \textbf{Seno (dispari):} $\sin(-\theta) = -\sin(\theta)$
    \item \textbf{Periodicità:} $\cos(\theta + 2k\pi) = \cos(\theta)$, $\sin(\theta + 2k\pi) = \sin(\theta)$ per $k \in \mathbb{Z}$.
\end{itemize}

\section{Proprietà Fondamentali di Concetti Statistici}

\subsection{Media (o Valore Atteso)}
La media, indicata con $E[X]$ o $m_X$, rappresenta il valore medio di una variabile casuale.

\begin{itemize}
    \item \textbf{Linearità:} Per due variabili casuali X e Y e due costanti a, b:
    $$ E[aX + bY] = aE[X] + bE[Y] $$
    
    \item \textbf{Media di una costante:} Il valore atteso di una costante è la costante stessa.
    $$ E[c] = c $$
    
    \item \textbf{Calcolo tramite PDF:} Per una variabile continua X con densità di probabilità $f_X(x)$:
    $$ E[X] = \int_{-\infty}^{\infty} x \cdot f_X(x) \,dx $$
\end{itemize}


\subsection{Varianza}
La varianza, indicata con $Var(X)$ o $\sigma_X^2$, misura la dispersione dei valori di una variabile casuale attorno alla sua media.

\begin{itemize}
    \item \textbf{Non-negatività:} La varianza è sempre maggiore o uguale a zero.
    $$ Var(X) \ge 0 $$
    
    \item \textbf{Varianza di una costante:} La varianza di una costante è zero.
    $$ Var(c) = 0 $$
    
    \item \textbf{Proprietà di scala:} Per una costante $a$ e una variabile X:
    $$ Var(aX + b) = a^2 Var(X) $$
    
    \item \textbf{Relazione con la media:} La varianza può essere calcolata come la differenza tra il momento secondo (la media del quadrato) e il quadrato della media.
    $$ Var(X) = E[X^2] - (E[X])^2 $$
    
    \item \textbf{Varianza della somma:} Per due variabili X e Y, la varianza della loro somma è:
    $$ Var(X + Y) = Var(X) + Var(Y) + 2\text{Cov}(X, Y) $$
    Se X e Y sono incorrelate, la loro covarianza è zero e la formula si semplifica in $Var(X + Y) = Var(X) + Var(Y)$.
\end{itemize}


\subsection{Potenza}
La potenza media di un processo stazionario $X(t)$ è il valore atteso del suo quadrato, che corrisponde al suo momento secondo.

\begin{itemize}
    \item \textbf{Relazione con media e varianza:} La potenza totale è la somma della potenza della componente continua (DC power), pari al quadrato della media, e della potenza della componente alternata (AC power), pari alla varianza.
    $$ P_X = E[X^2] = \sigma_X^2 + m_X^2 $$
    
    \item \textbf{Relazione con l'autocorrelazione:} La potenza è uguale alla funzione di autocorrelazione calcolata per un ritardo nullo.
    $$ P_X = R_X(0) $$
\end{itemize}


\subsection{Autocovarianza}
L'autocovarianza, $C_X(\tau)$, di un processo stazionario misura la covarianza tra il processo e una sua versione traslata nel tempo di un ritardo $\tau$.

\begin{itemize}
    \item \textbf{Relazione con l'autocorrelazione:} È legata all'autocorrelazione e alla media del processo.
    $$ C_X(\tau) = R_X(\tau) - m_X^2 $$
    
    \item \textbf{Valore per ritardo nullo:} Per un ritardo $\tau = 0$, l'autocovarianza coincide con la varianza del processo.
    $$ C_X(0) = \sigma_X^2 $$
    
    \item \textbf{Simmetria:} La funzione di autocovarianza è una funzione pari.
    $$ C_X(\tau) = C_X(-\tau) $$
    
    \item \textbf{Limite massimo:} Il suo valore massimo si ha sempre per ritardo nullo.
    $$ |C_X(\tau)| \le C_X(0) $$
\end{itemize}


\subsection{Autocorrelazione}
L'autocorrelazione, $R_X(\tau)$, di un processo stazionario misura la correlazione (cioè la somiglianza) tra il processo e una sua versione traslata di un ritardo $\tau$.

\begin{itemize}
    \item \textbf{Valore per ritardo nullo:} Per un ritardo $\tau = 0$, l'autocorrelazione coincide con la potenza media del processo.
    $$ R_X(0) = E[X^2] = P_X $$
    
    \item \textbf{Simmetria:} La funzione di autocorrelazione è una funzione pari.
    $$ R_X(\tau) = R_X(-\tau) $$
    
    \item \textbf{Limite massimo:} Il suo valore massimo (in modulo) si ha sempre per ritardo nullo.
    $$ |R_X(\tau)| \le R_X(0) $$
\end{itemize}


\subsection{Densità di Probabilità (PDF - Probability Density Function)}
La PDF, indicata con $f_X(x)$, descrive la probabilità relativa che una variabile casuale continua X assuma un determinato valore.

\begin{itemize}
    \item \textbf{Non-negatività:} La PDF è sempre maggiore o uguale a zero per ogni valore di x.
    $$ f_X(x) \ge 0 $$
    
    \item \textbf{Area unitaria:} L'area totale sottesa dalla curva della PDF è uguale a 1.
    $$ \int_{-\infty}^{\infty} f_X(x) \,dx = 1 $$
    
    \item \textbf{Calcolo della probabilità:} La probabilità che la variabile X cada in un intervallo $[a, b]$ si calcola integrando la PDF in quell'intervallo.
    $$ P(a \le X \le b) = \int_{a}^{b} f_X(x) \,dx $$
\end{itemize}

\section{Formulario di Integrali Utili}

\begin{table}[h!]
\centering
\caption{Formule Integrali}
\label{tab:integrals}
\begin{tabular}{ll}
    \toprule
    \textbf{Descrizione} & \textbf{Integrale Indefinito} \\
    \midrule
    \multicolumn{2}{l}{\textit{Integrali Fondamentali}} \\
    Potenza ($n \neq -1$) & $\int x^n dx = \frac{x^{n+1}}{n+1} + C$ \\
    Reciproco & $\int \frac{1}{x} dx = \ln|x| + C$ \\
    Costante & $\int k dx = kx + C$ \\
    \midrule
    \multicolumn{2}{l}{\textit{Funzioni Esponenziali e Logaritmiche}} \\
    Esponenziale (base e) & $\int e^x dx = e^x + C$ \\
    Esponenziale (base a) & $\int a^x dx = \frac{a^x}{\ln(a)} + C$ \\
    Logaritmo Naturale & $\int \ln(x) dx = x\ln(x) - x + C$ \\
    \midrule
    \multicolumn{2}{l}{\textit{Funzioni Trigonometriche}} \\
    Seno & $\int \sin(x) dx = -\cos(x) + C$ \\
    Coseno & $\int \cos(x) dx = \sin(x) + C$ \\
    Tangente & $\int \tan(x) dx = -\ln|\cos(x)| + C$ \\
    Secante al quadrato & $\int \sec^2(x) dx = \tan(x) + C$ \\
    \midrule
    \multicolumn{2}{l}{\textit{Funzioni Risultanti in Trigonometriche Inverse}} \\
    Arcoseno & $\int \frac{1}{\sqrt{1-x^2}} dx = \arcsin(x) + C$ \\
    Arcotangente & $\int \frac{1}{1+x^2} dx = \arctan(x) + C$ \\
    \bottomrule
\end{tabular}
\end{table}