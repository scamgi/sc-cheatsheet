%%%%%%%%%%%%%%%%%%%%%%%%%%%%%%%%%%%%%%%%%%%%%%%%%%%%%%%%
% chapters/formule_generali.tex
%%%%%%%%%%%%%%%%%%%%%%%%%%%%%%%%%%%%%%%%%%%%%%%%%%%%%%%%
\chapter*{Formulario Generale}
\addcontentsline{toc}{chapter}{Formulario Generale}

\begin{multicols}{2} % Dividiamo il formulario in due colonne

\section{Trasformata di Fourier}
\begin{definition}
Trasformata di Fourier (FT):
$$ X(f) = \FT\{x(t)\} = \int_{-\infty}^{+\infty} x(t) e^{-j2\pi f t} dt $$
Anti-trasformata di Fourier (IFT):
$$ x(t) = \IFT\{X(f)\} = \int_{-\infty}^{+\infty} X(f) e^{j2\pi f t} df $$
\end{definition}

\subsection*{Principali Coppie di Trasformate}
\begin{tabular}{ll}
\toprule
\textbf{Segnale $x(t)$} & \textbf{Trasformata $X(f)$} \\
\midrule
$\delta(t)$ & $1$ \\
$1$ & $\delta(f)$ \\
$\rect\left(\frac{t}{T}\right)$ & $T \sinc(fT)$ \\
$\sinc(Wt)$ & $\frac{1}{W}\rect\left(\frac{f}{W}\right)$ \\
$\tri\left(\frac{t}{T}\right)$ & $T \sinc^2(fT)$ \\
$e^{j2\pi f_0 t}$ & $\delta(f - f_0)$ \\
$\cos(2\pi f_0 t)$ & $\frac{1}{2}[\delta(f-f_0) + \delta(f+f_0)]$ \\
$\sin(2\pi f_0 t)$ & $\frac{1}{2j}[\delta(f-f_0) - \delta(f+f_0)]$ \\
\bottomrule
\end{tabular}

\subsection*{Proprietà della Trasformata di Fourier}
\begin{tabular}{ll}
\toprule
\textbf{Proprietà} & \textbf{Dominio del Tempo $\leftrightarrow$ Frequenza} \\
\midrule
Linearità & $ax_1(t)+bx_2(t) \leftrightarrow aX_1(f)+bX_2(f)$ \\
Traslazione & $x(t-t_0) \leftrightarrow X(f)e^{-j2\pi f t_0}$ \\
Modulazione & $x(t)e^{j2\pi f_0 t} \leftrightarrow X(f-f_0)$ \\
Scaling & $x(at) \leftrightarrow \frac{1}{|a|}X\left(\frac{f}{a}\right)$ \\
Convoluzione & $x(t) * h(t) \leftrightarrow X(f)H(f)$ \\
Prodotto & $x(t)h(t) \leftrightarrow X(f)*H(f)$ \\
Derivazione & $\frac{d}{dt}x(t) \leftrightarrow j2\pi f X(f)$ \\
Dualità & $X(t) \leftrightarrow x(-f)$ \\
\bottomrule
\end{tabular}

\section{Processi Casuali}
\begin{definition}
\textbf{Valor medio:} $m_x = E[x_n]$ \\
\textbf{Varianza:} $\sigma_x^2 = E[(x_n - m_x)^2]$ \\
\textbf{Potenza:} $P_x = E[|x_n|^2] = \sigma_x^2 + m_x^2$ \\
\textbf{Autocovarianza:} $C_x[m] = E[(x_{n+m}-m_x)(x_n-m_x)^*]$ \\
\textbf{Autocorrelazione:} $R_x[m] = E[x_{n+m}x_n^*] = C_x[m] + |m_x|^2$ \\
\end{definition}

\subsection*{Relazioni per Sistemi LTI}
Dato un processo $x_n$ in ingresso a un sistema LTI con risposta impulsiva $h_n$, il processo d'uscita $y_n$ ha:
\begin{itemize}
    \item \textbf{Valor medio:} $m_y = m_x \cdot H(0) = m_x \sum_k h_k$
    \item \textbf{DSP (PSD):} $S_y(f) = S_x(f) |H(f)|^2$
    \item \textbf{Potenza:} $P_y = \int_{-1/2}^{1/2} S_y(f) df = R_y[0]$
\end{itemize}

\subsection*{Teorema di Wiener-Khinchin}
La Densità Spettrale di Potenza (DSP) è la trasformata di Fourier della funzione di autocorrelazione.
$$ S_x(f) = \FT\{R_x[m]\} $$
$$ R_x[m] = \IFT\{S_x(f)\} $$

\section{Campionamento e DFT}
\begin{definition}
\textbf{Teorema del Campionamento:} $f_s \ge 2 f_{\max}$ per evitare aliasing.
\end{definition}
\textbf{Spettro del segnale campionato $x_n = x(nT)$:}
$$ \tilde{X}(f) = \frac{1}{T} \sum_{k=-\infty}^{\infty} X\left(f - \frac{k}{T}\right) = f_s \sum_{k=-\infty}^{\infty} X(f - k f_s) $$
\textbf{DFT (N campioni):}
$$ X_k = \sum_{n=0}^{N-1} x_n e^{-j\frac{2\pi}{N}kn} $$
\textbf{IDFT (N campioni):}
$$ x_n = \frac{1}{N} \sum_{k=0}^{N-1} X_k e^{j\frac{2\pi}{N}kn} $$

\end{multicols}