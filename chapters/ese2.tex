%%%%%%%%%%%%%%%%%%%%%%%%%%%%%%%%%%%%%%%%%%%%%%%%%%%%%%%%
% chapters/ese2.tex
%%%%%%%%%%%%%%%%%%%%%%%%%%%%%%%%%%%%%%%%%%%%%%%%%%%%%%%%
\chapter{Esercizio 2: Campionamento e DFT}

\section{Teorema del Campionamento}
\begin{itemize}
    \item \textbf{Condizione di Nyquist (no aliasing):} $f_s \ge 2B$, dove $B$ è la massima frequenza (di banda) del segnale.
    \item \textbf{Intervallo di campionamento massimo:} $T_{\max} = \frac{1}{2B}$
    \item \textbf{Relazione Frequenza e Frequenza Normalizzata:} $f = \varphi \cdot f_s$
\end{itemize}

\section{Trasformata di Fourier del Segnale Campionato (DTFT)}
\begin{itemize}
    \item La trasformata $\tilde{X}(f)$ del segnale campionato è la somma delle repliche periodiche della trasformata originale $X(f)$, scalata per $f_s$:
    \[
        \tilde{X}(f) = f_s \sum_{k=-\infty}^{\infty} X(f - k f_s)
    \]
    \item In \textbf{frequenza normalizzata} $\varphi = f/f_s$:
    \[
        \tilde{X}(\varphi) = \sum_{k=-\infty}^{\infty} X(f_s(\varphi - k))
    \]
\end{itemize}

\section{Segnale Ricostruito}
\begin{itemize}
    \item La trasformata del segnale ricostruito $x_R(t)$ si ottiene filtrando il segnale campionato con un filtro passa-basso ideale di banda $f_s/2$:
    \[
        X_R(f) = \frac{1}{f_s} \tilde{X}(f) \cdot \rect\left(\frac{f}{f_s}\right)
    \]
\end{itemize}

\section{Trasformata Discreta di Fourier (DFT) su N campioni}
\begin{itemize}
    \item \textbf{Analisi (DFT):} $X_k = \sum_{n=0}^{N-1} x_n e^{-j2\pi \frac{kn}{N}}$
    \item \textbf{Sintesi (IDFT):} $x_n = \frac{1}{N} \sum_{k=0}^{N-1} X_k e^{j2\pi \frac{kn}{N}}$
    \item \textbf{Convoluzione Circolare:} $z_n = x_n \circledast y_n \quad \iff \quad Z_k = X_k \cdot Y_k$
    \item \textbf{DFT di un impulso traslato:} $x_n = A \delta_{n-n_0} \quad \iff \quad X_k = A e^{-j2\pi \frac{kn_0}{N}}$
    \item \textbf{DFT di una costante:} $x_n = A \quad \iff \quad X_k = A \cdot N \cdot \delta_k$
    \item \textbf{DFT di un coseno:} $x_n = \cos\left(2\pi \frac{k_0 n}{N}\right) \quad \iff \quad X_k = \frac{N}{2}[\delta_{k-k_0} + \delta_{k-(N-k_0)}]$
    \item \textbf{DFT di un seno:} $x_n = \sin\left(2\pi \frac{k_0 n}{N}\right) \quad \iff \quad X_k = \frac{N}{2j}[\delta_{k-k_0} - \delta_{k-(N-k_0)}]$
\end{itemize}

